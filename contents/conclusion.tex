% !TEX root = ../main.tex
\chapter{Conclusion and Outlook}

The performance of the measurement of the CP mixing angle $\alpha$ in the fermionic coupling of 
the Higgs boson to the top quark was presented in this thesis exploiting the effective coupling of the Higgs boson to gluons. 
Therefore, the analysis of the search for the decay of the Higgs boson into a pair of tau leptons \cite{Sirunyan:2017khh} was adapted to investigate the gluon-gluon fusion production process of the Higgs boson.
A categorization was developed leading to a larger number of selected Higgs bosons produced via gluon-gluon fusion in the signal categories compared to VBF produced Higgs bosons.
In the same step, it was taken care to achieve a close-to-optimal signal-to-background ratio.
Furthermore, the background estimation was tuned by implementing the data-driven \textit{simultaneous equation method} in the semileptonic decay channels of the tau lepton pair.
With the cut-based categorization of $\mathsf{gg\rightarrow H+2j}$ events used in this approach, a measurement, based on a pseudo-data set constructed from the background and the SM Higgs boson prediction,
was performed for two sets of CP sensitive observables - $D_{0-}$, $D_\text{CP}$ and $\Delta\phi_\text{jj}$. \newline{}
For the final fit systematic uncertainties covering biases in the Higgs boson plus jets cross sections were added and groups of jet energy correction uncertainties were formed in addition to the uncertainties taken from Ref. \cite{Sirunyan:2017khh}. 
Using the azimuthal angular difference $\Delta\phi_\text{jj}$ between the two initial-state jets, $\alpha$ is measured with a precision of $\text{45}\degree$. A scalar and a pseudoscalar coupling can be separated at $\text{1.25\,\sigma}$. Exploiting the full 
kinematic information in a matrix element likelihood approach with optimal observables a better performance is obtained with a precision of $\text{41}\degree$. Here, a separation between scalar and pseudoscalar scenarios of $\text{1.36\,\sigma}$ is achieved.

During the evaluation of the analysis setup, the impact of a potential CP violating contribution in the vector boson coupling of the Higgs boson was studied. It was proven that CP violation in VBF does not affect the measurement of the fermionic coupling.
Moreover, the model-dependent performance of the analysis was investigated. For this, the signal strength of the fermionic coupling was set to the SM prediction $\mu_\text{F}=1$. 
It is noted that the analysis is then capable to discriminate scalar and pseudoscalar hypotheses by $\text{3.65\,\sigma}$ with an accuracy of $\text{24}\degree$. This measurement 
comes with the assumption that no other BSM effects modify the rate of the effective interaction between the gluons and the Higgs boson.

In order to improve the performance quoted here several further step could be executed. 
The cut-based approach developed for this thesis only provides a close-to-optimal sensitivity. In the latest 
CMS publications of the observation of the Higgs boson decaying into bottom quarks \cite{hbb} and in the associated production with top quarks \cite{Sirunyan:2312113} modern machine learning techniques proved to 
be ideal choices for an optimal event categorization. A similar approach is currently developed for this analysis, too.
On the technical side, the background estimation can be further improved implementing new techniques such as the fake-factor method and using embedded $\mathsf{Z\rightarrow \tau\tau}$ samples \cite{Bechtel:48943}.
Last but not least, it has been shown that the precision of the measurement is limited by statistical fluctuations. Thus, a combination with the full Run 2 data set
is expected to improve the performance.
The analysis presented in this thesis provides a simple setup that is sensitive to the CP information in fermionic couplings of the Higgs boson produced via gluon-gluon fusion.
The approach looks promising as it already now can measure the CP mixing angle to an accuracy of roughly $\text{41}\degree$.
As soon as, the analysis may be unblinded, it will provide the first measurement of this CP property in data with the CMS experiment.
